\documentclass[12pt]{article}
\usepackage{natbib,amsmath,amsfonts,fullpage,hyphenat,booktabs,graphicx,setspace}
\usepackage[colorlinks,linkcolor=blue,citecolor=blue,urlcolor=blue]{hyperref}
\onehalfspacing

\title{Project Proposal\\Robot Manipulation}
\author{Hossam Mohamed \\ Vajra Ganesh \\Dharmin Bakaraniya}
\begin{document}
\maketitle{}
\section{Introduction}
\subsection{Team name}
Manipulator is fit.
\subsection{Project title}
\textbf{Optimal Inverse Reachability}
\subsection{Names of members}
\begin{table}[htpb]
        \centering
        \begin{tabular}{|c|c|}
                \hline
                \textbf{Name} & \textbf{Git Id}\\\hline
                Dharmin Bakaraniya & DharminB\\\hline
                Vajra Ganeshkumar & vajrag\\\hline
                Hossam Mohammed & Hibrahim1\\\hline
        \end{tabular}
\end{table}

\section{An introduction about our project}
Given an area with objects that need to be manipulated, the project will find out which
is the best position for the base so that the biggest number of
objects are contained in the dexterous workspace of the
youbot.


\section{Available software packages (or available approaches)}
\begin{itemize}
        \item Reuleaux\cite{makhal}\cite{reuleaux}
        \item Extend move\_base\_to\_manip\cite{movebasetomanip}
        \item Implementing our own wrapper on \textit{Moveit!} using concepts of \cite{malek} \cite{park} or \cite{rastegar}
\end{itemize}
\section{A small qualitative comparison}
\begin{table}[htpb]
        \centering
        \caption{Qualitative Analysis}
        \label{tab:qualAna}
        \begin{tabular}{|c|c|c|}\hline
                \textbf{Reuleaux} & \textbf{Extention} & \textbf{Our Wrapper}\\\hline
                C++ & Python & Python\\\hline
                moveit\_ikfast & moveit & moveit \\\hline
        %\hline
        \end{tabular}
\end{table}
\section{2 Best approach}
\begin{enumerate}
        \item Reuleaux
        \item Extend move\_base\_to\_manip
\end{enumerate}
\section{Experimental methodology}
% DOUBT
\section{Work plan}
\subsection{What needs to be done}
\begin{itemize}
        \item Figure out Reuleaux package and how to use it
        \item Figure out move\_base\_to\_manip and how to use it
        \item Build an extension for move\_base\_to\_manip
        \item Run these in simulation
        \item Run these on real youbot
\end{itemize}
\subsection{By when does it need to be done}
By the end of June.
\subsection{Who is the responsible person for it}
All the team members.
\begin{thebibliography}{1}

\bibitem{malek}Abdel-Malek, Karim, Wei Yu, and Jingzhou Yang. "Placement of robot manipulators to maximize dexterity." International Journal of Robotics and Automation 19.1 (2004): 6-14.
\bibitem{park}Park, F.C.; Brockett, R.W., “Kinematic dexterity of robotic mechanisms”, International
Journal of Robotics Research, v 13 n 1 Feb 1994 p 1-15
\bibitem{makhal}Makhal, Abhijit, and Alex K. Goins. "Reuleaux: Robot Base Placement by Reachability Analysis." 2018 Second IEEE International Conference on Robotic Computing (IRC). IEEE, 2018.
\bibitem{rastegar}Rastegar, J.; Singh, J.R., 1994, “New probabilistic method for the performance evaluation
of manipulators”, ASME Journal of Mechanical Design, v 116 n 2, pp. 462-466.
\bibitem{rosquestion}\href{https://answers.ros.org/question/261304/what-packages-apart-from-reuleaux-can-be-used-for-optimal-base-positioning-of-a-robot-for-mobile-manipulation/}{Ros forum question for inverse reachability} 
\bibitem{movebasetomanip}\href{http://wiki.ros.org/move_base_to_manip}{ros move\_base\_to\_manip}
\bibitem{reuleaux}\href{http://wiki.ros.org/reuleaux}{ros reuleaux}
\end{thebibliography}
\end{document}
